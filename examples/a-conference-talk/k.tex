% Copyright 2007 by Till Tantau
%
% This file may be distributed and/or modified
%
% 1. under the LaTeX Project Public License and/or
% 2. under the GNU Public License.
%
% See the file doc/licenses/LICENSE for more details.



\documentclass{beamer}

%
% DO NOT USE THIS FILE AS A TEMPLATE FOR YOUR OWN TALKS!!
%
% Use a file in the directory solutions instead.
% They are much better suited.
%


% Setup appearance:

\usetheme{Darmstadt}
\usefonttheme[onlylarge]{structurebold}
\setbeamerfont*{frametitle}{size=\normalsize,series=\bfseries}
\setbeamertemplate{navigation symbols}{}


% Standard packages

\usepackage[english]{babel}
\usepackage[latin1]{inputenc}
\usepackage{times}
\usepackage[T1]{fontenc}

\usepackage{fontspec}
\usepackage{xeCJK}       
\usepackage{graphicx}

% Setup TikZ

\usepackage{tikz}
\usetikzlibrary{arrows}
\tikzstyle{block}=[draw opacity=0.7,line width=1.4cm]


% Author, Title, etc.

\title[Title] 
{
  Abstract:~~~
}

\author[Gramm, Hartman, Nierhoff, Sharan, Tantau]
{
  Author1\inst{1} \and
  Author2\inst{2} \and
  \textcolor{green!50!black}{Till~TantauAuthor3}\inst{3}
}

\institute[Institute Name]
{
  \inst{1}%
  University 1, Germany
  \and
  \vskip-2mm
  \inst{2}%
  University 2, Israel
  \and
  \vskip-2mm
  \inst{3}%
  University 3, Germany
}

\date[WABI 2014]
{2014.3.23}

% The main document

\begin{document}
\setromanfont{SimSun}

\begin{frame}
  \titlepage
\end{frame}

\begin{frame}{Outline}
  \tableofcontents
\end{frame}


\section{Introduction}

\subsection{The Model and the Problem}

\begin{frame}{What is haplotyping and why is it important?}
  You hopefully know this after the previous three talks\dots
\end{frame}

\begin{frame}[t]{General formalization of haplotyping.}
  \begin{block}{Inputs}
    \begin{itemize}
    \item A \alert{genotype matrix} $G$.
    \item The \alert{rows} of the matrix are \alert{taxa / individuals}.
    \item The \alert{columns} of the matrix are \alert{SNP sites /
        characters}. 
    \end{itemize}
  \end{block}
  \begin{block}{Outputs}
    \begin{itemize}
    \item A \alert{haplotype matrix} $H$.
    \item Pairs of rows in $H$ \alert{explain} the rows of $G$.
    \item The haplotypes in $H$ are \alert{biologically plausible}. 
    \end{itemize}
  \end{block}
\end{frame}

\section{Bad News: Hardness Results}

\subsection{Hardness of PP-Partitioning of Haplotype Matrices}

\section*{Summary}

\begin{frame}
  \frametitle<presentation>{Summary}
  \begin{itemize}
  \item
    Finding optimal pp-partitions is \alert{intractable}. 
  \item
    It is even intractable to find a pp-partition when \alert{just two 
      noncontiguous  blocks are known to suffice}.
  \item
    For perfect \alert{path} phylogenies, optimal partitions can be
    computed \alert{in polynomial time}.
  \end{itemize}
\end{frame}


\appendix
\section*{Appendix}
%appendix frames
\end{document}


