\documentclass{beamer}
% AnnArbor  Antibes  Bergen  Berkeley  Berlin  Boadilla  CambridgeUS  Copenhagen  Darmstadt  Dresden  Frankfurt  Goettingen  Hannover  Ilmenau  JuanLesPins  Luebeck  Madrid  Malmoe  Marburg  Montpellier  PaloAlto  Pittsburgh  Rochester  Singapore  Szeged  Warsaw  boxes  default 
\mode<presentation>
    {
      \usetheme{Berkeley}
      \setbeamercovered{transparent}
    }
    \usepackage[english]{babel}
    \usepackage{graphicx}           
    \usepackage{xeCJK}       
    \usepackage{fontspec}
    \usepackage{courier}
    
    \title[A Case Study in SoF] {A Case Study in Stakeholder-oriented Goal-modeling Framework}
          
    \subtitle{}
              
    \author[Author]{Jipeng Wu \and Eryu Ding \and Bin Luo}
                     
     \institute[Universities of Somewhere and Elsewhere] {
                                 Software Institute\\
                                 Nanjing University
     }
                               
\date[ICSESS 2014]
{ICSESS Presentations, 2014}

\subject{Theoretical Computer Science}
\AtBeginSubsection[]
{
  \begin{frame}<beamer>{Outline}
    \tableofcontents[currentsection,currentsubsection]
  \end{frame}
}

%10~12 minutes oral presentation
\begin{document}

\begin{frame}
  \titlepage
\end{frame}

\begin{frame}{Outline}
  \tableofcontents
\end{frame}

\section{Motivation}
\subsection{The Basic Problem That We Studied}
\begin{frame}{Why We Need SoF}              
  \footnotesize{
  \begin{itemize}
  \item
    We applied goal methods in a RE process.
    \begin{itemize}
    \item Goal Methods are
      \begin{itemize}
      \item diverse, but discrete and fragmental  
      \item rely on a certain context 
      \end{itemize}
    \item
      Goal methods includes: goal modeling, goal specification, goal analysis(formal or informal, validation, verification, elaboration, conflict management)
    \item 
      We do need these methods, they are the core of goal-based RE, but they are not enough to compose a complete RE process.
    \item 
      RE process, more precisely speaking, the early phase of RE process is our research target.
    \item
      "It is unwise to apply goal-based requirements methods in isolation". [Potts, 1997] 
    \item RE process 
      \begin{itemize}
      \item consistent and monolithic
      \item not reply on a certain context.   Because it is context itself.
      \end{itemize}
    \end{itemize}
  \end{itemize}
  }
\end{frame}
\begin{frame}{Why We Need SoF}          %[2]
  \begin{itemize}
  \item
    To guide a RE process, we presented a possible solution——SoF(Stakeholder-oriented Goal Modeling Framework).
  \item 
    On this abstraction level, the most important RE concerns are:
    \begin{itemize}
    \item Goal Methods Selection-->is about How to integrate proper goal methods.
    \item Goal Indentification-->How to identify initial goals.
    \item Goal Validation-->is about How to ensure that a correct goal reasoning result accords with true requirements of stakeholders. 
    \end{itemize}
  \end{itemize}
\end{frame}

\begin{frame}{Features of SoF}             %[3]
  \begin{itemize}
  \item Initial goals acquition couples goals and scenarios together
    \begin{enumerate}
    \item to convene scenario-based interviews with stakeholders
    \item a structured scenario description text in nature languge to organize the interview results
    \end{enumerate}
  \item Goal modeling is the basis of elaboration reasoning  and stakeholder-involved validation. It combines KAOS goal model with RWS annotated goal tree model together.
  \item Less important details
    \begin{enumerate}
    \item goal specification 
    \item design of validation interviews
    \end{enumerate}
  \end{itemize}
\end{frame}

\subsection{Previous Work}             %[4]
\begin{frame}{KAOS}
  \begin{itemize}
  \item
    KAOS [van Lamsweerde, 1995]
    \begin{enumerate}
    \item
      Although KAOS is a complete RE approach, we are concerned only with its goal model. 
    \item
      KAOS goal model defines some meta-concepts—goal, action, agent, entity and event, which can be visualized as nodes.
    \item 
      The edges between nodes capture the semantic links between such abstractions.
      \begin{enumerate}
      \item Two basic link types——AND/OR.
      \item Extended link types: Contributes(+), ContributesStrongly(++), Conflicts(-), and ConflictsStrongly(--).
      \end{enumerate}
    \end{enumerate}
  \end{itemize}
\end{frame}
\begin{frame}{RWS}                     %[5]
  \begin{itemize}
  \item
    Real World Scenes [Haumer, 1998] , a scenario-based approach
    \begin{enumerate}
    \item
      Current system should be captured in the form of rich media(e.g., taking photos, recording videos). The observation results are called Real World Scene. 
    \item 
      The observation results should be linked to goals, in order to elaborate and validate goals in the follow-up work.
    \item
      RWS annotated the goal model with views of stakeholders(1.agree, 2.not agree, 3.add more goals and 4.no position), which facilitates review and validation and finally conforms the goal model to the real world scene. 
    \end{enumerate}
  \end{itemize}
\end{frame}

\section{Our Proposal and Case Study}
\subsection{SoF Process}

\begin{frame}{SoF Elaboration Activity}
  \footnotesize{
  \begin{enumerate}
  \item
    SoF combines requirements acquition, requirements elaboration reasoning and requirements validation as one atomic activity, which is called \emph{SoF Elaboration Activity}.
  \item
    Each successful \emph{SoF Elaboration Activity} includes the following phases: 
    \begin{enumerate}
    \item interview with stakeholders, updating \emph{"Scenario Description"}
    \item elaborating goal models 
    \item validation interview
    \end{enumerate}
  \item
    Before all the steps of the \emph{SoF Elaboration Activity} of one requirement have been finished, it is not allowed that the \emph{SoF Elaboration Activity} of another requirement is initiated.
  \item
    SoF introduces a mechanism against requirements change. It allows stakeholders to send change requests at any time in the SoF process, which is the only legal way to interrupt SoF Elaboration Activities.
  \end{enumerate}
  }
\end{frame}

\begin{frame}{Activity Diagram of SoF Process}
  \begin{figure}
    \includegraphics[width=0.6in]{img/2_0.PNG}
    \caption{Activity Diagram of SoF Process}
  \end{figure}
\end{frame}  

\begin{frame}{Detailed Activity Diagram of SoF Process}
  \begin{figure}
    \includegraphics[width=0.6in]{img/2.PNG}
    \caption{Activity Diagram of SoF Process}
  \end{figure}
  SoF adopts a traditional way, conducting a interview, to acquire requirements. It is insufficient to rely only on stakeholders' knowledge and judgment. Thus it is necessary to find a way to specify the description of scenarios of future system from stakeholders and help stakeholders to obtain a consensus.
\end{frame}  

\begin{frame}{Detailed Activity Diagram of SoF Process}
  \begin{figure}
    \includegraphics[width=0.6in]{img/2_1.PNG}
    \caption{Activity Diagram of SoF Process}
  \end{figure}
  The first initial interview generates a brief structured specification, describing high-level goals in the form of scenarios. This specification is updated after requirements acquisition phase of each \emph{SoF Elaboration Activity}. Each update is a basis of subsequent goal refinement and goal validation.
\end{frame}  

\begin{frame}{Detailed Activity Diagram of SoF Process}
  \begin{figure}
    \includegraphics[width=0.6in]{img/2_2.PNG}
    \caption{Activity Diagram of SoF Process}
  \end{figure}
  \footnotesize{
\begin{enumerate}
\item The goal model adopted by SoF is named after \emph{SoF Annotated Goal Tree Model}. It is based on KAOS and extended by introducing annotations for stakeholder evaluation.
\item  Because SoF Elaboration Activity is atomic, SoF Annotated Goal Tree Model and Structured Scenario Description reaches the same level of completeness and relevance, which is the precondition of Elaboration Reasoning and Validation.
\item  In addition, only the goals that have passed stakeholder evaluation are allowed to be a basis of elaboration, the atomicity of SoF Elaboration Activity also ensures the quality of elaborations.
\end{enumerate}
}
\end{frame}

\subsection{Structured Scenario Description}
\begin{frame}{Why is a Scenario Description Needed?}
  \begin{itemize}
  \item
    A summary of stakeholder interview.  It records stakeholders' expectations of the future system.
  \item
    A readable document for stakeholders.   Scenarios are used to organize complex requirements.  
  \item
    A basis of subsequent goal refinement and goal validation. 
  \end{itemize}
\end{frame}

\begin{frame}{How to Write and Maintain a Scenario Description?}
  \footnotesize{
  \begin{itemize}
  \item
    There are various and unlimited ways to write a scenario description. 
    \begin{enumerate}
    \item Formal or Informal
    \item In Nature Language or Algebraic Language.
    \item Flat text or specific data structure.
    \end{enumerate}
    The grammar and structure of such structured description can be defined by requirements developers. The main purpose is to organize readable documents from requirements fragments in accordance with their scenarios for the convenience of modification of subsequent work and requirements evaluation.
  \end{itemize}
  }
\end{frame}

\begin{frame}{Initial Interviews Related Work}
  \begin{enumerate}
  \item Questions designed by RE developers.
    The initial interview poses questions about the background, problem domain, workflow of all possible scenarios of the future system.
  \item Doucumentation 
        Requirements developers are responsible for combining information fragments obtained from the interview together to generate a structured specification.
  \item Maintainance 
        In the subsequent \emph{SoF Elaboration Activities}, more details will be added in the structured specification of scenarios.
  \end{enumerate}
\end{frame}

\begin{frame}{Example of a Possible Implementation of Scenario Description}
  \begin{figure}
    \includegraphics[width=0.5in]{img/3.PNG}
    \caption{Structured Scenario Description}
  \end{figure}
    This description specifies possible events, trigger events of these events and concrete actions of stakeholders in these events in nature languge.
    It is a specification of a scenario which have passed one \emph{SoF Elaboration Activity} and have been enriched with some details.
\end{frame}

\begin{frame}{Example of a Possible Implementation of Scenario Description}
  \begin{enumerate}
  \item
    In this example, stakeholders describe the scenario: "how a technical manager manage projects."
  \item 
    This scenario includes four events:
    \begin{enumerate}
    \item \emph{create a project}
    \item \emph{select development team}
    \item \emph{edit project info}
    \item \emph{and query project progress}
    \end{enumerate}
    details of these events are specified in the form of triggering events and action description.
  \item Informal but Informative,
    \begin{enumerate}
    \item It used in goal validation interviews because its readability.
    \item Goal elaboration should not directly use this informal description, but the information provided by it helps the reasoning of \emph{SoF Annotated Goal Tree Model}.
    \item 
     For example, they can decompose the high-level goal, \emph{improve efficiency of project management}, to several child goals supporting this goal since they have the knowledge of what contributes to the efficiency of project management. E.g, to create a project more efficiently.
    \end{enumerate}
  \end{enumerate}
\end{frame}  


\subsection{SoF Annotated Goal Tree}
\begin{frame}{Design of Annotated Goal Tree Model}
  \begin{enumerate}
  \item Form:
    \begin{enumerate}
    \item KAOS Goal Model  (top-down decomposed)
    \item RWS Annotation  (for stakeholder validation)
    \end{enumerate}
  \item A goal reasoning tool: 
    \begin{enumerate}
    \item Goal Refinement Reasoning
    \item Goal Confict Management
    \item Requirements Evaluation
    \end{enumerate}
  \item A documentation of goals
  \item A communication material in validation interviews
  \end{enumerate}
  \begin{enumerate}
  \item   The annotations can be used to organize more structured evaluation interviews and actively engage stakeholders in such interviews.
  \item   After high-level goals determined in an initial interview, the subsequent activities, such as goal elicitation, conflict resolution and requirements evaluation, will be executed based on \emph{SoF Annotated Goal Tree Model}.
  \end{enumerate}
\end{frame}  
\begin{frame}{2 types of Goal Tree Models}
  \begin{enumerate}
  \item Pass the Validation Interview: a basis of the next \emph{SoF Elaboration Activity}
  \item Not Pass the Interview: a driven model for communication with stakeholders and control of requirements changes.
  \item Goal validation phase of \emph{SoF Elaboration Activity} can be further decomposed according to the design of annotations of \emph{SoF Annotated Goal Tree Model}.
    \begin{enumerate}
    \item relevance validation
    \item success validation
    \end{enumerate}
  \item
   -->We designed a comparatively simpler annotation in this case because there are no complex stakeholder constituents or serious interest conflicts. The annotation of \emph{SoF Annotated Goal Tree Model} can be modified if more complex evaluation process design is required.
  \end{enumerate}
\end{frame}

\begin{frame}{\emph{SoF Annotated Goal Tree Model} adds the following two types of marks:}            
  \begin{itemize}
  \item
    The mark "relevance" is used to record whether a goal has passed relevance validation.
  \item 
    The evaluation is based on stakeholders perspectives and latest structured specification based on scenarios.
  \item 
    If the goal is irrelevant to the future system, it is marked with No, otherwise it is marked with Yes.
  \item 
    Goals cannot enter the next step of success validation until it passes the relevance validation.
  \item
    The mark "agreed" is used to record whether a relevant goal has passed success validation.
  \item 
    The evaluation is based on stakeholders perspectives on the practical significance, constraints and cost of the evaluated goal.
  \item 
    If a it does not agree with stakeholders' expectations, the goal should be still denied.
  \end{itemize}
\end{frame}


\begin{frame}  {Goal Elaboration-->Image!!}
  \begin{figure}
    \includegraphics[width=0.8in]{img/4.PNG}
    \caption{SoF Annotated Goal Tree(Validated,Level 1 Elaboration)}
  \end{figure}
  \tiny{
  \begin{enumerate}
  \item  In Fig.3 we presents a high-level goal $G_1$—"To make technical department manager make better decisions when build project teams".
  \item  By asking Why/How questions we get child goals supporting $G_1$: $G_{1.1}$ and $G_{1.2}$.
    \begin{enumerate}
    \item  $G_{1.1}$ is "To provide better support of information on developers and project managers for techinical department manager".
    \item  $G_{1.2}$ is "To provide a mechanism allowing developers to reply with feedback to the manager's decisions".
    \end{enumerate}
  \item  These two goals were validated as child goals supporting their parent goal in different aspects, and thus successfully passed relevance validation. Stakeholders agreed that these two goals are consistent with their expectations. The goal tree was allowed to be further refined after each goal had passed both validations.
  \end{enumerate}
  }
\end{frame}

\begin{frame} {Further Elaboration}
  \begin{figure}
    \includegraphics[width=0.6in]{img/5.PNG}
    \caption{SoF Annotated Goal Tree(Validated,Level 2 Elaboration)}
  \end{figure}
  In this Fig, we get a complete \emph{SoF Annotated Goal Tree Model} that has been further elaborated. To emphasize the grammar,  we omit detailed description of each goal in Fig.4 and represent each goal only with its indexed symbol.
\end{frame}   

\begin{frame}  {Further Elaboration}   %[16]
  \begin{itemize}
  \item
    Here lists the detailed description of each goal that is not mentioned above.
    \begin{enumerate}
    \item $G_{1.1.1}$ is "Project managers input information of developers". 
    \item $G_{1.1.2}$ is "To access information of developers". 
    \item $G_{1.1.3}$ is "Project managers should update information of developers periodically". 
    \item $G_{1.1.1}$、$G_{1.1.2}$、$G_{1.1.3}$ work together to support their parent goal.
    \end{enumerate}
    If any of them fails to pass the evaluation, the whole elaboration plan fails.
  \end{itemize}
\end{frame}

\begin{frame} {Further Elaboration}
  \begin{itemize}
  \item
    \begin{enumerate}
    \item $G_{1.2.1}$ is "To provide an instant messaging platform for technical department managers, project managers and developers".
    \item  $G_{1.2.2}$ is "To publish decisions made by technical department managers and allow developers and project managers to reply asynchronously".
    \item $G_{1.2.1}$ does not conflict any other goal, and facilitates $G_{1.1}$ because the establishment of communication platform contributes positively to technical department managers' knowledge of developers' information. Thus $G_{1.2.1}$ passed both validations.
    \item $G_{1.2.2}$, although had passed the relevance validation, however, failed to pass the success validation because stakeholders believe that asynchronous communication is not practical and efficient enough to ensure the timeliness and richness of feedbacks.      
    \end{enumerate}
  \end{itemize}
\end{frame}

 



\section*{Summary}
\begin{frame}{Summary}
  \tiny{
  \begin{itemize}
  \item
    Against requirements nondeterminism.-->Acquire more complete and relevant requirements in practice of projects with requirements nondeterminism.
  \item
    Ensuring the quality of each step of elaboration.-->The atomicity of SoF Elaboration Activity also ensures the quality of elaborations.
  \item 
    Consistency between goal models and stakeholders' conceptual models.-->Each step of SoF activities is stakeholder-centered, which ensures that stakeholders' description of future system consists with goal models. With such consistency, SoF provides a reasonable context for KAOS Goal reasoning.
  \end{itemize}
  }
  \vskip0pt plus.5fill
        \tiny{
          \begin{itemize}
  \item
    Outlook
    \begin{itemize}
    \item
      Difficulty in validating the effectiveness of my proposal.
    \item
      Cost Problem.
      \begin{enumerate}
      \item The development cost is relatively higher because recursively executing SoF Elaboration Activities results in considerably frequent communication between and among requirements developers and stakeholders
      \item collection and management of raw data from stakeholders
      \item ubiquitous involvement of stakeholders -->We should decide whether a project is adaptive in applying SoF methods at the beginning of each project. Some projects with relatively constant and determinate requirements are not supposed to apply SoF methods.
      \end{enumerate}
    \end{itemize}
  \end{itemize}
                }

\end{frame}

\end{document}


